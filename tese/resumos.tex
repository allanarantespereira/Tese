%=========================================================================================================
% Resumo

\begin{singlespace}
\begin{center}
\section*{RESUMO GERAL}
\end{center}

Uma das vantagens dos modelos de regressão não linear é ter
interpretação para os parâmetros. Em muitas situações, parâmetros de
interesse, expressos como função dos parâmetros do modelo, são
quantidades sujeitas à investigação. Surge então a preocupação de como
fazer inferência sobre eles. Para isso, o método delta, a simulação
Monte Carlo e procedimentos bootstrap são alternativas
frequentes. Além disso, uma reparametrização pode ser aplicada ao
modelo de forma à representar tais parâmetros de interesse. Além de
melhorar a interpretação, a presença do parâmetro alvo estende as
possibilidades com relação a especificação de modelos e inferência
estatística. O objetivo com esse trabalho é sistematizar o
procedimento de aplicar reparametrizações. Ênfase foi dada em modelos
não lineares considerados em aplicações dentro das Ciências
Agrárias. Uma lista com 17 modelos reparametrizados é fornecida. No
primeiro estudo de caso, o nível de dano econômico da desfolha no
algodoeiro foi avaliado com os seguintes objetivos: 1) propor uma
parametrização de modelo que representasse o nível de dano econômico,
2) avaliar parametrizações alternativas por meio de suas propriedades,
onde considerando medidas de não linearidade, 3) aplicar inferência
baseada em verossimilhança, 4) selecionar um modelo para descrever a
relação entre produção e desfolha do algodoeiro em função do estágio
fenológico. O modelo reparametrizado apresentou melhores propriedades
nos estágios fenológicos com pronunciada relação não linear. No
restante, as medidas de curvatura, as correlações dos estimadores e os
gráficos de perfil de verossimilhança indicaram que um sub-modelo
deveria ser considerado. No segundo estudo de caso, objetiva-se
verificar o efeito da posição de amostragem e profundidade do solo
sobre os parâmetros $I$ (\emph{infletion}) e $S$ (\emph{slope}) da
curva de retenção de água do solo. Para isso 1) considerou-se ANOVA
simples e 2) ANOVA ponderada pela variância das estimativas desses
parâmetros em cada unidade experimental em comparação com 3) o uso de
modelos não lineares de efeitos mistos em uma parametrização
desenvolvida. Nenhum dos métodos alternativos de análise foi superior
ao modelo não linear de efeitos mistos na parametrização desenvolvida,
que apresentou intervalos de confiança mais estreitos para os
parâmetros e apontou efeito de posição e profundidade de coleta.\\
\newline
\noindent {Palavras-chave:} Verossimilhança. Método delta. Medidas de
curvatura. Efeitos mistos. van Genuchten.
\end{singlespace}

%=========================================================================================================
% Abstract

\newpage
\begin{singlespace}
\begin{center}
\section*{GENERAL ABSTRACT}
\end{center}

One of the advantages of the nonlinear regression models is to have
interpretable parameters. In many instances, the parameters of
interest, expressed as a function of the model parameters, are
quantities subject to investigation. Then comes the concern of how to
make inferences about them. For this, the delta method, the Monte
Carlo simulation and bootstrap procedures are common alternatives. In
addition, a reparametrization can be applied to the model in order to
represent these parameters of interest into the model. In addition to
improving the interpretation of the presence of the target parameter
extends the possibilities regarding the specification of models and
statistical inference. The aim of this work is to systematize the
procedure to apply reparametrizations. Emphasis was given on nonlinear
models considered in applications within the Agricultural Sciences. A
list with 17 models reparametrized is provided. In the first case
study, the threshold level of defoliation on cotton was evaluated with
the following objectives: 1) to propose a model parameter that
represents the level of economic damage, 2) evaluate alternative
parameterizations through its properties, which considering measures
of nonlinearity, 3) apply inference based on likelihood, 4) select a
model to describe the relationship between yield and defoliation of
cotton in each phenological stage. The reparametrized model showed
better properties in phenological stages with pronounced nonlinear
relationship. Otherwise the measures of curvature, the correlations of
the estimators and likelihood profile plots indicated that a sub-model
should be considered. In the second case study, the objective is to
verify the effect of sampling position and soil depth on the
parameters $I$ (\emph{infletion}) and $S$ (\emph{slope}) of the soil
water retention curve. For that 1) it was considered ANOVA and 2)
weighted ANOVA in each experimental unit compared to 3) using
nonlinear mixed effects on a parameterization developed. None of the
alternative methods of analysis was superior to model nonlinear mixed
effects in the parameterization developed, which had narrower
confidence intervals for the parameters and pointed sampling position
and depth effect.\\
\newline
\noindent {Keywords:} Likelihood. Delta method. Curvature measures. Mixed
effects. van Genuchten.

\end{singlespace}
