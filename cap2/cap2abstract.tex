Water is essential for crop production because it is involved in the
transport of nutrients, chemical reactions and physical processes
maintaining the soil life. The knowledge about the water retention
curve (WRC) of the soil is important to establish management
strategies. The physical quality of the soil is dependent on the WRC
and the parameters $I$, the inflection point of the WRC, and $S$, the
rate of change at the inflection point, considered as indicators of
quality physical parameters are related to descriptive measures the
pore size distribution of the soil. In this work, we aim to
investigate the effect of sampling position and soil depth on the
parameters $I$ and $S$ of the WRC. For that 1) it was considered ANOVA
and 2) weighted ANOVA based on the variance of these estimates in each
experimental unit compared to 3) using nonlinear mixed-effects in a
parameterization developed for $I$ and $S$. None of the alternative
methods of analysis was superior to the nonlinear mixed effects model
in the parameterization developed, which showed narrower intervals for
parameter estimates and pointed sampling position and depth effect on
parameters $I$ and $S$.\\
\newline
\noindent {Key-words}: Reparametrization. Function of parameters. Delta method. Water retention curve.
