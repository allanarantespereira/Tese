The effect of defoliation on the quality and productivity of crops is
essential information to define management strategies, such as
intensity and frequency of grazing and harvesting and the
establishment of economic threshold in order to aid decisions about
controlling defoliating pests. For the cotton, as well as many others
crops, reduction of production by defoliation can be represented by a
nonincreasing monotone function. Several models can satisfy this
restriction, however, there is concern about inferring the economic
damage level, $\vartheta_q$, by adjusting a model. Yield-defoliation
data of cotton due to the phenological stage are considered to infer
about the economic damage level with the following objectives: 1) to
propose a model that represents the parameter $\vartheta_q$, 2)
evaluate alternative parameterizations through measures of
nonlinearity, 3) apply inference based on likelihood, 4) select a
model to describe the relationship between yield and defoliation of
cotton in each phenological stage. The reparametrized model had lower
measures of nonlinearity in phenological stages with pronounced
nonlinear relationship. In the others, the measures of curvature, the
correlations of the estimators and likelihood profile plots indicated
that a sub-model should be considered.\\
\newline
\noindent {Key-words}: Parameter interpretation. Lokelihood. Delta method. Curvature. \emph{Gossypium hirsutum}.
