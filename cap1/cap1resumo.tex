Modelos de regressão não linear são considerados quando existe algum
conhecimento preliminar sobre a relação entre variáveis. Tal
conhecimento pode ser a respeito da própria natureza dos dados, uma
equação diferencial e até mesmo a forma do diagrama de dispersão entre
as variáveis. Em geral, seus parâmetros têm interpretação. Além disso,
parâmetros de interesse, expressos como função dos parâmetros do
modelo, são alvos de investigação. Para isso, o método delta,
simulação Monte Carlo e procedimentos bootstrap são procedimentos
adotados para fazer inferência. Além disso, uma reparametrização pode
ser aplicada ao modelo de forma a representar esses parâmetros de
interesse. Além de melhorar a interpretação do modelo, a presença do
parâmetro alvo estende as possibilidades com relação a especificação
de modelos e inferência estatística.  O objetivo com esse trabalho é
sistematizar o procedimento de aplicar reparametrizações. Ênfase é
dada em modelos não lineares considerados em Ciências Agrárias. Uma
lista com 17 modelos reparametrizados é fornecida. Breve discussão
sobre os métodos de inferência é feita.\\
\newline
\noindent {Palavras-chave}: Função de parâmetros. Interpretação de parâmetros. Verossimilhança. Método delta. Curvatura.
