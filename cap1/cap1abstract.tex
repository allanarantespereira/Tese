Nonlinear regression models are considered when there is some prior
knowledge of the relationship between variables. Such knowledge can be
about the nature of the data, a differential equation, and even the
shape of the scatter diagram between the variables. In general, the
parameters have an interpretation. Furthermore, parameters of
interest, expressed as a function of the model parameters, are targets
of investigation. For this, the delta method, Monte Carlo simulation
and bootstrap procedures are procedures used to make inferences. In
addition, a reparametrization can be applied to the model to represent
the parameters of interest. In addition to improving the
interpretation of the model, the presence of the target parameter
extends the possibilities regarding the specification of models and
statistical inference. The aim of this work is to systematize the
procedure to apply reparametrizações.  Emphasis is on nonlinear models
considered in Agricultural Sciences.  A list with 17 models
reparametrized is provided. Brief discussion on the methods of
inference is made.\\
\newline
\noindent {Key-words}: Function of parameters. Parameter 
interpretation. Lokelihood. Delta Method. Curvature.
